\documentclass{article}
\usepackage{xspace}
\usepackage[utf8]{inputenc}
\usepackage[automark]{scrlayer-scrpage}
\usepackage[T1]{fontenc}
\usepackage{graphicx}
\usepackage{multirow}
\usepackage{array}
\usepackage{booktabs}
\usepackage{caption}
\usepackage{siunitx}
\usepackage{xcolor}
\usepackage{makecell}
\usepackage{tabularx}
\usepackage{listings}
\usepackage{algorithm}
\usepackage{algpseudocode}

\renewcommand{\arraystretch}{1.4}
\setlength{\tabcolsep}{4pt}

  \sisetup{
        scientific-notation = true,
    exponent-mode = input,
    round-mode = places,
    round-precision = 2,  % Changed from 1 to 2
    group-digits = false,
    tight-spacing = true,
    table-format = 1.2e+1,  % Changed from 1.5 to 1.2
    table-number-alignment = center,
    table-auto-round = true
  }

\lstset{
  language=C,
  frame=single,
  basicstyle=\ttfamily\footnotesize
}

% Macro for MPI_Allgather. Use it for typesetting MPI_Allgather: \mpiallgather
\newcommand{\mpiallgather}{\texttt{MPI\_\-Allgather}\xspace}
% You can add similar macros here for better typesetting

\title{HPC Project Report 2025}
\author{Student Names}
\date{\today}

\begin{document}
\maketitle

\section{First Section}

It is a good idea to use sections to structure your report. Use
Table~\ref{tab:performance_comparison} as a template for your reporting your
findings.

%%%%%%%%%%%%%%%%%%%%%%%%%%%% RESULTS TABLE %%%%%%%%%%%%%%%%%%%%%%%%%%%%%%%%%%%%
\begin{table}
  \centering
  \caption{Performance of algorithms (execution time in microseconds). Configuration
    with  \num{20} Nodes $\times$ \num{10} PPN.}
  \label{tab:performance_comparison}
  \scriptsize
  \begin{tabular}{
    |l|
    c|
    S[table-format=1.2e+1]|  % Consistent format for all columns
    S[table-format=1.2e+1]|
    S[table-format=1.2e+1]|
    S[table-format=1.2e+1]|
    S[table-format=1.2e+1]|
    S[table-format=1.2e+1]|
  }
    \hline
    \multirow{2}{*}{\textbf{Algorithm}} &
    \multirow{2}{*}{\textbf{Type}} &
    \multicolumn{6}{c|}{\textbf{Message size (number of elements)}} \\
    \cline{3-8}
    & & {\textbf{1}} & {\textbf{10}} & {\textbf{100}} & {\textbf{1000}} & {\textbf{10000}} & {\textbf{100000}} \\
    \hline
% You can paste your results here
\multirow{3}{*}{Baseline}
& 0  & 2.00e+00 & 1.40e+00 & 5.60e+00 & 3.82e+01 & 3.61e+02 & 3.67e+03 \\
\cline{2-8}
& 1  & 1.00e+00 & 2.70e+00 & 5.00e+00 & 3.69e+01 & 2.92e+02 & 2.67e+03 \\
\cline{2-8}
& 2  & 9.00e-01 & 1.40e+00 & 6.40e+00 & 2.86e+01 & 2.79e+02 & 2.70e+03 \\
\hline
\multirow{3}{*}{Bruck}
& 0  & 1.80e+00 & 1.40e+00 & 5.00e+00 & 4.10e+01 & 3.61e+02 & 3.61e+03 \\
\cline{2-8}
& 1  & 1.00e+00 & 1.30e+00 & 4.70e+00 & 2.89e+01 & 2.49e+02 & 2.63e+03 \\
\cline{2-8}
& 2  & 2.90e+00 & 1.00e+00 & 4.40e+00 & 2.70e+01 & 2.47e+02 & 2.62e+03 \\
\hline
\multirow{3}{*}{Circulant}
& 0  & 1.40e+00 & 1.20e+00 & 4.40e+00 & 1.81e+01 & 1.49e+02 & 1.81e+03 \\
\cline{2-8}
& 1  & 1.00e+00 & 1.20e+00 & 4.80e+00 & 2.54e+01 & 2.83e+02 & 2.54e+03 \\
\cline{2-8}
& 2  & 9.00e-01 & 3.00e+00 & 3.90e+00 & 2.45e+01 & 3.77e+02 & 2.57e+03 \\
    \hline
% End of your results
  \end{tabular}
\end{table}
%%%%%%%%%%%%%%%%%%%%%%%%%%%%%%%%%%%%%%%%%%%%%%%%%%%%%%%%%%%%%%%%%%%%%%%%%%%%%%%

\end{document}
